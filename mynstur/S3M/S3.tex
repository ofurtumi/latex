\documentclass{article}
\usepackage[utf8]{inputenc}
\usepackage[icelandic]{babel}
\usepackage[T1]{fontenc}
\usepackage{mathtools}
\usepackage{amsmath}
\usepackage{amssymb}

\title{Skilaverkefni 3 - Stærðfræðimynstur}
\author{ttb3@hi.is}
\date{\today}

\begin{document}
\maketitle

\section*{Verkefni 1}
Forritið tekur inn þessi gildi:
\begin{list}{}{}
    \item mkk = meðallaun karla fyrri skatt
    \item mkvk = meðallaun kvenna fyrri skatt
    \item ke = kyn einstaklings
    \item me = mánaðarlaun einstaklings
\end{list}
og athugar hvort me sé hærra eða lægra en tilheyrandi meðallaun, þ.e.

\subsection*{a)}
Bakmengi fallsins eru öll þau gildi sem myndmengi fallsins getur valið úr. Í þessu tilfelli er það output forritsins, þ.e. 

\begin{center}
    $\mathbb{Y} \equiv \{$"Yfir meðallaunum", "Undir meðallaunum"$\}$
\end{center} 

\subsection*{b)}
Formengið er öll stök sem fallið getur tekið inn, það eru

\begin{align*}
    \mathbb{X} = (mkk,mkvk,me,ke|mkk,mkvk,me\in \mathbb{R} \land ke \in \{"kona","karl" \})
\end{align*} 

\newpage
\section*{Verkefni 2}
Með fylkjamargföldun fáum við tvær yrðingar
\begin{align*}
    (a\land b)\lor(\lnot b\land T)=T\\
    (b\land b)\lor(b\land T)=\lnot a
\end{align*}
byrjum að vinna með seinni yrðinguna, notum \emph{sjálfsvalsregluna} vinstra megin við eðun og \emph{samsemdarregluna} hægra megin við eðun
útkoman úr því er 

\begin{align*}
    b\lor b=\lnot a
\end{align*} 

og þá er hægt að nota \emph{sjálfsvalsregluna} á þá yrðingu og fá út að 
\begin{align*}
    b=\lnot a
\end{align*}

snúum okkur þá að fyrri yrðingunni og stingum $\lnot a$ inn í staðinn fyrir $b$, yrðingin lítur þá svona út
\begin{align*}
    (a\land \lnot a)\lor(\lnot b\land T)=T
\end{align*}
þá er hægt að nota \emph{samsemdarregluna} hægra megin og \emph{neitunarregluna} vinstra megin til þess að fá
\begin{align*}
    F\lor\lnot b=T
\end{align*}
þá beitir maður örsnöggri \emph{samsemdarreglu} og er kominn með
\begin{align*}
    \lnot b = T
\end{align*}
þá er hægt að setja upp lista yfir gildi
\begin{align*}
    \lnot b&= T\\
    b&=F\\
    b&=\lnot a\\
    \lnot a&=F\\
    a&=T
\end{align*}
Þá eru gildin á a og b þessi $a=1$ og $b=0$

\newpage
\section*{Verkefni 3}
Í hvert skipti sem maður þvær á sér hendurnar drepur maður 99.9\% af bakteríum, það skilur eftir 0.1\% baktería eða 0.001 sinnum upprunalega fjölda baktería. 
Þannig er hægt að finna út hversu oft þarf að þvo sér um hendur með því að leysa fyrir veldið n yfir 0.001 og það má sýna með þessari formúlu
\begin{align}
    10^9*0.001^n&=1\\
    0.001^n&=\frac{1}{10^9}\\
    log(0.001^n)&=log(\frac{1}{10^9})\\
    n*log(0.001)&=log(\frac{1}{10^9})\\
    n&=\frac{log\frac{1}{10^9}}{log(0.001)}\\
    n&=3
\end{align}
Maður þarf að þvo sér þrisvar sinnum um hendurnar til þess að enda með aðeins 1 bakteríu á hvorri hendi.

\newpage
\section*{Verkefni 4}
\subsection*{a)}
\begin{align*}
    \sum_{k=0}^{n}12k(k+4)&=\sum_{k=0}^{n}12k^2+48k\\
    &=12\sum_{k=0}^{n}k^2+12\sum_{k=0}^{n}4k\\
    &=12\frac{n(n+1)(2n+1)}{6}+48\frac{n(n+1)}{2}
\end{align*}
\subsection*{b)}
\begin{align*}
    \sum_{k=1}^{n}4k^2(6+k)&=\sum_{k=1}^{n}24k^2+4k^3\\
    &=24\sum_{k=1}^{n}k^2+4\sum_{k=1}^{n}k^3\\
    &=24\frac{n(n+1)(2n+1)}{6}+4\frac{n^2(n+1)^2}{4}
\end{align*}
Þar sem að útkoman á þessu falli minnkar hratt eftir því sem k stækkar, og k stækkar endalaust, er hægt að segja að útkoman úr þessu falli sé 0
\begin{align*}
    \sum_{k=2}^\infty0.5^k
\end{align*}
\begin{align*}
    0.5^2&=\frac{1}{4}\\
    0.5^3&=\frac{1}{8}\\
    0.5^4&=\frac{1}{16}\\
    0.5^5&=\frac{1}{32}\\
    0.5^{10}&=\frac{1}{1024}\\
    0.5^{20}&=\frac{1}{1048576}\\
    0.5^{25}&=\frac{1}{33554432}
\end{align*}
þessi veldisvöxtur heldur áfram og þegar k er komið upp í stærri tölur eins og 329 eða meir hætta reiknivélar að geta sýnt úttakið á nokkurn merkingalegan máta og sýna í staðinn bara 0

\end{document}