\documentclass{article}
\usepackage[utf8]{inputenc}
\usepackage[icelandic]{babel}
\usepackage[T1]{fontenc}
\usepackage{graphicx}
\usepackage{mathtools}
\usepackage{amsmath}
\usepackage{amssymb}
\usepackage{minted}


\graphicspath{ {./} }
\title{Æfing 1 - TÖV202}
\author{ttb3@hi.is}
\date{\today}


\begin{document}
\maketitle


\section*{Hluti 1}
\begin{center}
    \begin{tabular}{|c|c|c|}
        \hline
        A & B & out\\
        \hline
        0 & 0 & 0\\
        \hline
        1 & 0 & 0\\
        \hline
        1 & 1 & 1\\
        \hline
        0 & 1 & 1\\
        \hline
    \end{tabular}
\end{center}
\begin{center}
    \begin{tabular}{|c|c|c|c|c|}
        \hline
        A&B&C&D&out\\
        \hline
        0&0&0&0&0\\
        \hline
        0&0&0&1&0\\
        \hline
        0&0&1&0&1\\
        \hline
        0&0&1&1&1\\
        \hline
        0&1&0&0&0\\
        \hline
        0&1&0&1&0\\
        \hline
        0&1&1&0&0\\
        \hline
        0&1&1&1&1\\
        \hline
        1&0&0&0&1\\
        \hline
        1&0&0&1&0\\
        \hline
        1&0&1&0&1\\
        \hline
        1&0&1&1&1\\
        \hline
        1&1&0&0&1\\
        \hline
        1&1&0&1&0\\
        \hline
        1&1&1&0&1\\
        \hline
        1&1&1&1&1\\
        \hline
    \end{tabular}
\end{center}

\section*{Hluti 2}
\begin{center}
    \begin{tabular}{|c|c|c|c|c|}
        \hline
        a&b&c&outVinstri&outHægri\\
        \hline
        0&0&0&0&0\\
        \hline
        0&0&1&1&1\\
        \hline
        0&1&0&0&0\\
        \hline
        0&1&1&1&1\\
        \hline
        1&0&0&0&0\\
        \hline
        1&0&1&1&1\\
        \hline
        1&1&0&1&1\\
        \hline
        1&1&1&1&0\\
        \hline
    \end{tabular}
\end{center}

\section*{Hluti 3}
\begin{center}
    \begin{tabular}{|c|c|c|c|c|}
        \hline
        A&B&C&outStutt&outLong\\
        \hline
        0&0&0&0&0\\
        \hline
        0&0&1&0&0\\
        \hline
        0&1&0&1&1\\
        \hline
        0&1&1&1&1\\
        \hline
        1&0&0&1&1\\
        \hline
        1&0&1&0&0\\
        \hline
        1&1&0&1&1\\
        \hline
        1&1&1&1&0\\
        \hline

    \end{tabular}
\end{center}

\section*{Greinagerð}
Tilgangur verkefnisins var að rifja upp sannleikstöflur og læra aðeins á forritið sem átti að nota. 
Forritið sem varð fyrir valinu, þar sem cedar virkar ekki linux, var logisim. Logisim er öflugt forrit sem býður upp á 
að fá sannleikstöflur í output og að láta notandann setja inn sínar eigin boolean jöfnur.

Ég teiknaði upp allar rásirnar og bar síðan útkomuna úr þeim við útkomuna sem gefin er í logisim. Þær voru í langflestum tilfellum
réttar hjá mér en þurfti aðeins að leiðrétt í hluta 3.

Það var áhugavert að læra á þetta forrit og ég hefði alveg verið til í að læra þetta fyrir síðustu önn þar sem það hefði
komið sér ansi vel þá líka.

Sjá má niðurstöður fyrir ofan

\end{document}