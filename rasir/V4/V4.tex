\documentclass{article}
\usepackage[utf8]{inputenc}
\usepackage[icelandic]{babel}
\usepackage[T1]{fontenc}
\usepackage{graphicx}
\usepackage{mathtools}
\usepackage{amsmath}
\usepackage{amssymb}
\usepackage{minted}


\graphicspath{ {./} }
\title{Verkefni 4 - ghsr}
\author{ttb3@hi.is}
\date{\today}

\begin{document}
\maketitle


\section*{um hvað snérist verkefnið?}
verkefnið fólst í því að útbúa rökrás sem getur tekið við einni 3 bita tölu og einni 2 bita tölu, margfaldað þær saman og skilað útkomunni.
útkoman er táknuð með 5 bita tölu og getur í tíundarkerfi að hámarki verið 21. 
það þurfti að nota nýja tegund af töflum til að finna jöfnur rásarinnar, bitamargföldunartöflu??, veit ekki alveg hvað þær heita samt.
verkefnið lét líka nemanda nota adders í praktík sem hjálpar til við að skilja þá betur.

\section*{Hvað gerði ég}
ég byrjaði á því að reikna binary margföldun tveggja og þriggja bita tölu þar sem báðar tölur voru eins stórar og þær gátu verið, þ.e. $111$ og $11$.
ég gerði þennan útreikning á blað og hreinskrifaði svo upp í excel og fékk þetta út:
\begin{center}
    \includegraphics{imgs/Screenshot from 2022-03-11 10-17-35.png}
\end{center}

núna veit ég hvar carrybitar koma fram og get skrifað upp bitamargföldunartöflu, hvað heitir þessi týpa af töflum??, 
gerði það eins og með útreikningin, fyrst á blað svo í excel, svona endaði taflan:
\begin{center}
    \includegraphics[scale=0.55]{imgs/Screenshot from 2022-03-11 10-23-10.png}
\end{center}

núna veit ég hvernig boolean jafna fyrir rás hvers og eins output lítur út og þá er voða lítið mál að teikna það upp í tölvunni:
\begin{center}
    \includegraphics[scale=0.4]{imgs/Screenshot from 2022-03-11 10-51-59.png}
\end{center}

\newpage
síðasta skrefið var svo bara að láta forritið herma rásina fyrir mig og athuga hvort hún sé rétt, sanntaflan er huge og lítur svona út:
\begin{center}
    \begin{tabular}{|c|c|c|c|c|c|c|c|c|c|}
        \hline
        A0&A1&A2&B0&B1&P4&P3&P2&P1&P0\\
        \hline
        0&0&0&0&0&0&0&0&0&0\\
        \hline
        0&0&0&0&1&0&1&0&0&0\\
        \hline
        0&0&0&1&0&0&0&0&0&0\\
        \hline
        0&0&0&1&1&0&1&0&0&0\\
        \hline
        0&0&1&0&0&0&0&0&0&0\\
        \hline
        0&0&1&0&1&0&1&0&0&0\\
        \hline
        0&0&1&1&0&0&0&1&0&0\\
        \hline
        0&0&1&1&1&0&1&1&0&0\\
        \hline
        0&1&0&0&0&0&0&0&0&0\\
        \hline
        0&1&0&0&1&0&1&1&0&0\\
        \hline
        0&1&0&1&0&0&0&0&1&0\\
        \hline
        0&1&0&1&1&0&1&1&1&0\\
        \hline
        0&1&1&0&0&0&0&0&0&0\\
        \hline
        0&1&1&0&1&0&1&1&0&0\\
        \hline
        0&1&1&1&0&0&0&1&1&0\\
        \hline
        0&1&1&1&1&1&0&0&1&0\\
        \hline
        1&0&0&0&0&0&0&0&0&0\\
        \hline
        1&0&0&0&1&0&1&0&1&0\\
        \hline
        1&0&0&1&0&0&0&0&0&1\\
        \hline
        1&0&0&1&1&0&1&0&1&1\\
        \hline
        1&0&1&0&0&0&0&0&0&0\\
        \hline
        1&0&1&0&1&0&1&0&1&0\\
        \hline
        1&0&1&1&0&0&0&1&0&1\\
        \hline
        1&0&1&1&1&0&1&1&1&1\\
        \hline
        1&1&0&0&0&0&0&0&0&0\\
        \hline
        1&1&0&0&1&0&1&1&1&0\\
        \hline
        1&1&0&1&0&0&0&0&1&1\\
        \hline
        1&1&0&1&1&1&0&0&0&1\\
        \hline
        1&1&1&0&0&0&0&0&0&0\\
        \hline
        1&1&1&0&1&0&1&1&1&0\\
        \hline
        1&1&1&1&0&0&0&1&1&1\\
        \hline
        1&1&1&1&1&1&0&1&0&1\\
        \hline
    \end{tabular}
\end{center}

\section*{hvernig gekk?}
verkefnið gekk vel um leið og ég skildi bitamargföldunartöfluna en ég þurfti að horfa á fyrirlesturinn tvisvar til þess að skilja þær.
verkefnið var skemmtilegt og rúllaði vel áfram. ég þurfti að beita meiri rökhugsun frekar en fræðilegri hugsun og mér finnst það alltaf gaman.

\section*{niðurstöður}
\begin{center}
    \includegraphics[scale=0.55]{imgs/Screenshot from 2022-03-11 10-23-10.png}
\end{center}
loka bitamargföldunartaflan

\begin{center}
    \includegraphics[scale=0.4]{imgs/Screenshot from 2022-03-11 10-51-59.png}
\end{center}
loka rökrás

\begin{center}
    \begin{tabular}{|c|c|c|c|c|c|c|c|c|c|}
        \hline
        A0&A1&A2&B0&B1&P4&P3&P2&P1&P0\\
        \hline
        0&0&0&0&0&0&0&0&0&0\\
        \hline
        0&0&0&0&1&0&1&0&0&0\\
        \hline
        0&0&0&1&0&0&0&0&0&0\\
        \hline
        0&0&0&1&1&0&1&0&0&0\\
        \hline
        0&0&1&0&0&0&0&0&0&0\\
        \hline
        0&0&1&0&1&0&1&0&0&0\\
        \hline
        0&0&1&1&0&0&0&1&0&0\\
        \hline
        0&0&1&1&1&0&1&1&0&0\\
        \hline
        0&1&0&0&0&0&0&0&0&0\\
        \hline
        0&1&0&0&1&0&1&1&0&0\\
        \hline
        0&1&0&1&0&0&0&0&1&0\\
        \hline
        0&1&0&1&1&0&1&1&1&0\\
        \hline
        0&1&1&0&0&0&0&0&0&0\\
        \hline
        0&1&1&0&1&0&1&1&0&0\\
        \hline
        0&1&1&1&0&0&0&1&1&0\\
        \hline
        0&1&1&1&1&1&0&0&1&0\\
        \hline
        1&0&0&0&0&0&0&0&0&0\\
        \hline
        1&0&0&0&1&0&1&0&1&0\\
        \hline
        1&0&0&1&0&0&0&0&0&1\\
        \hline
        1&0&0&1&1&0&1&0&1&1\\
        \hline
        1&0&1&0&0&0&0&0&0&0\\
        \hline
        1&0&1&0&1&0&1&0&1&0\\
        \hline
        1&0&1&1&0&0&0&1&0&1\\
        \hline
        1&0&1&1&1&0&1&1&1&1\\
        \hline
        1&1&0&0&0&0&0&0&0&0\\
        \hline
        1&1&0&0&1&0&1&1&1&0\\
        \hline
        1&1&0&1&0&0&0&0&1&1\\
        \hline
        1&1&0&1&1&1&0&0&0&1\\
        \hline
        1&1&1&0&0&0&0&0&0&0\\
        \hline
        1&1&1&0&1&0&1&1&1&0\\
        \hline
        1&1&1&1&0&0&0&1&1&1\\
        \hline
        1&1&1&1&1&1&0&1&0&1\\
        \hline
    \end{tabular}
\end{center}
loka sanntafla

\end{document}