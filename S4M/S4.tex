\documentclass{article}
\usepackage[utf8]{inputenc}
\usepackage[icelandic]{babel}
\usepackage[T1]{fontenc}
\usepackage{enumitem}
\usepackage{amsmath,mathtools,amssymb}
\usepackage{algorithmicx,algpseudocodex}

\title{Skilaverkefni 4 - Stærðfræðimynstur}
\author{ttb3@hi.is, gbt6@hi.is, dal17@hi.is}
\date{\today}

\begin{document}
\maketitle

\section*{Verkefni 1}
\subsection*{a)}

\begin{algorithmic}
\State function SUMMA(n):
\State $x \gets 0$
\For{$i \gets 1 \ to \ n$}
    \State $x \gets x + i^2$
\EndFor
\State return x
\end{algorithmic}

\subsection*{b)}

\begin{algorithmic}
    \State function FIBONOTCCI(n)
    \If{$n\leq 1$}
        \State return 1
    \Else 
        \State return $2(n-1)+3(n-2)$
    \EndIf
\end{algorithmic}

\newpage
\section*{Verkefni 2}
\subsection*{a)}
Leysum fyrir $\lim_{x\to\infty} \frac{f(x)}{g(x)}<\infty$, $f(x)$, $g(x)$ og stóra $O$ eru eftirfarandi:
\begin{align*}
    O(x)\\
    g(x)&=x\\
    f(x)&=\frac{x^4+x^2+1}{x^3}
\end{align*}
    
Leysum:
\begin{align*}
    \lim_{x\to\infty} \frac{f(x)}{g(x)} 
    &= \frac{\frac{x^4+x^2+1}{x^3}}{x}\\
    &= \frac{x^4+x^2+1}{x^4}\\
    &= \frac{\frac{x^4}{x^4}+\frac{x^2}{x^4}+\frac{1}{x^4}}{\frac{x^4}{x^4}}\\
    &= \frac{1+\frac{1}{x^2}+\frac{1}{x^4}}{1}\\
    &= 1+\frac{1}{x^2}+\frac{1}{x^4}\\
    &\to 1+0+0\\
    &= 1\\
    1 &< \infty
\end{align*}

\subsection*{b)}
Notum sömu aðferð til að sanna, $f(x)$, $g(x)$ og stóra $O$ eru eftirfarandi:
\begin{align*}
    O(1)\\
    g(x)&=x\\
    f(x)&=(1+\frac{1}{x^2})(1+\frac{5}{x^3})
\end{align*}

Leysum: 
\begin{align*}
    \lim_{x\to\infty} \frac{f(x)}{g(x)}
    &=\frac{(1+\frac{1}{x^2})(1+\frac{5}{x^3})}{1}\\
    &=(1+\frac{1}{x^2})(1+\frac{5}{x^3})\\
    &\to(1+0)(1+0)\\
    &=1\\
    1&<\infty
\end{align*}

\subsection*{c)}
Notum sömu aðferð til að sanna en nýtum reglu l'Hospital til þess að leysa út, $f(x)$, $g(x)$ og stóra $O$ eru eftirfarandi:

\begin{align*}
    O(1)\\
    g(x)&=1\\
    f(x)&=\frac{log(2x)}{log(x)}
\end{align*}

Leysum:
\begin{align*}
    \lim_{x\to\infty} \frac{f(x)}{g(x)}
    &=\frac{log(2x)}{log(x)}\\
    (Notum l'Hospital)&=\frac{\frac{1}{x}}{\frac{1}{x}}\\
    &=\frac{1}{x*\frac{1}{x}}\\
    &=1\\
    &\to1\\
    1&<\infty\\
\end{align*}

\section*{Verkefni 3}
\begin{align*}
   O(nlog(n))\ &nr.\ 7\\
   O(n^{-1/2})\ &nr.\ 6\\
   O(log(n))\ &nr.\ 9\\
   O(\pi^n)\ &nr.\ 3\\
   O(n)\ &nr.\ 8\\
   O(n^\pi)&nr.\ 5\\
   O(n!)\ &nr.\ 1\\
   O(n^{1000})\ &nr.\ 4\\
   O(5^n)\ &nr.\ 2
\end{align*}

\newpage
\section*{Verkefni 4}
\subsection*{a)}
Ef stafur í leitarstreng er italic passar hann ekki við viðmiðunarstreng, ef stafur í leitarstreng er bold passar hann við staf í viðmiðunarstreng

\begin{center}
    \begin{tabular}{c c c c c c c c}
        |\textit{F}|&|E|&| \ |&| \ |&| \ |&| \ |&| \ | & \ s=0\\
        |C|&|O|&|V|&|F|&|E|&|F|&|E| & \ j=1\\
        \\
        | \ |&|\textit{F}|&|E|&| \ |&| \ |&| \ |&| \ | & \ s=1\\
        |C|&|O|&|V|&|F|&|E|&|F|&|E| & \ j=1\\
        \\
        | \ |&| \ |&|\textit{F}|&|E|&| \ |&| \ |&| \ | & \ s=2\\
        |C|&|O|&|V|&|F|&|E|&|F|&|E| & \ j=1\\
        \\
        | \ |&| \ |&| \ |&|\textbf{F}|&|\textbf{E}|&| \ |&| \ | & \ s=3\\
        |C|&|O|&|V|&|F|&|E|&|F|&|E| & \ j=2\\
        \\
        | \ |&| \ |&| \ |&| \ |&|F|&|E|&| \ | & \ s=4\\
        |C|&|O|&|V|&|F|&|E|&|F|&|E| & \ j=1\\
        \\
        | \ |&| \ |&| \ |&| \ |&| \ |&|\textbf{F}|&|\textbf{E}| & \ s=5\\
        |C|&|O|&|V|&|F|&|E|&|F|&|E| & \ j=2\\
    \end{tabular}
\end{center}

\subsection*{b)}
Fyrst að fyrsti stafur í p er ekki til í t þá er s það eina sem hækkar og það þarf ekki að fara í while loopið, tímalykkjan er þá lengd for lykkjunnar eða O(n).

\newpage
\section*{Verkefni 5}
Til að finna hvort tekur fleiri aðgerðir þarf að leggja saman fjölda margföldunaraðgerða og samlagningaraðgerða fylkjanna og bera þær saman fyrir (AB)C og A(BC)
\begin{align*}
    (AB)C 
    &=5*5*4 &&\text{Fjöldi margfaldana A og B}\\
    &+3*4*4 &&\text{Fjöldi samlagnina A og B}\\
    &+3*4*6 &&\text{Fjöldi margfaldana AB og C}\\
    &+3*3*6 &&\text{Fjöldi samlagnina AB og C}\\
    &=234 &&\text{Heildarfjöldi aðgerða}
\end{align*}

\begin{align*}
    A(BC) 
    &=5*4*6 &&\text{Fjöldi margfaldana B og C}\\
    &+5*4*6 &&\text{Fjöldi samlagnina B og C}\\
    &+3*5*6 &&\text{Fjöldi margfaldana BC og A}\\
    &+3*4*6 &&\text{Fjöldi samlagnina BC og A}\\
    &=372 &&\text{Heildarfjöldi aðgerða}
\end{align*}

Þannig er hægt að sjá að (AB)C tekur færri aðgerðir en A(BC) og er þessvegna hraðara

\section*{Verkefni 6}
Hægt er að nota sömu aðferð og notuð var í verkefni tvö til þess að sýna fram á að stóra O í $n^{10}2^n$ er $O(n^2)$ og þá er hægt að sjá auðveldlega að $10^n>2^n$

\end{document}