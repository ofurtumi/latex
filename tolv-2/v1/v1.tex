\documentclass{article}
\usepackage[utf8]{inputenc}
\usepackage[icelandic]{babel}
\usepackage[T1]{fontenc}
\usepackage{graphicx}
\usepackage{mathtools}
\usepackage{amsmath}
\usepackage{amssymb}
\usepackage{minted}


\graphicspath{ {./} }
\title{Titill - Áfangi}
\author{ttb3@hi.is}
\date{\today}


\begin{document}
\maketitle


\section*{Verkefni 1.3.3}
\subsection*{a)}
a er hægt að útfæra með eftirfarandi push/pop aðgerðum með því að fylgja eftirfarandi uppskrift:
$01234-----56789-----$

\subsection*{b)}
b er ekki hægt að útfæra með push/pop aðgerðum þar sem $901$ getur ekki átt sér stað:
$01234-56-78-----9-$ hér kemur villan í ljós, það væri hægt að enda á $910$ en ekki $901$

\subsection*{c)}
c er hægt að útfæra með eftirfarandi push/pop aðgerðum með því að fylgja eftirfarandi uppskrift:
$012-345-6-7--8-9---$

\subsection*{d)}
d er hægt að útfæra með eftirfarandi push/pop aðgerðum með því að fylgja eftirfarandi uppskrift:
$01234-----5-6-7-8-9-$

\subsection*{e)}
e er hægt að útfæra með eftirfarandi push/pop aðgerðum með því að fylgja eftirfarandi uppskrift:
$01-2-3-4-5-6-789----$

\subsection*{f)}
f er ekki hægt að útfæra með push/pop aðgerðum vegna þess að ekki er hægt að fá út $817$, það hefði þurft að vera búið að poppa $7$ og $2$ til þess að gera þetta mögulegt:
$0-1234-56---78-$ svo er ekki hægt að fara lengra

\subsection*{g)}
g er ekki hægt að útfæra með push/pop aðgerðum á sama hátt og b er ekki hægt, $302$ gengur ekki upp en $320$ hefði gengið upp:
$01-234-567-89-----$ svo er ekki hægt að fara lengra

\subsection*{h)}
h er hægt að útfæra með eftirfarandi push/pop aðgerðum með því að fylgja eftirfarandi uppskrift:
$012--34--56--78--9--$


\section*{Verkefni 1.3.8}
\begin{center}
    \begin{tabular}{|c|c|c|c|}
        \hline
        N&Lengd&Strengur&Breyting\\
        \hline
        0&1&it&\\
        \hline
        1&1&was&L*2\\
        \hline
        0&2&-&\\
        \hline
        1&2&the&\\
        \hline
        2&2&best&L*2\\
        \hline
        1&4&-&L/2\\
        \hline
        2&2&of&L*2\\
        \hline
        3&4&times&\\
        \hline
        2&4&-&\\
        \hline
        1&4&-&L/2\\
        \hline
        0&2&-&\\
        \hline
        1&2&it&\\
        \hline
        2&2&was&L*2\\
        \hline
        1&4&-&L/2\\
        \hline
        2&2&the&L*2\\
        \hline
        1&4&-&L/2\\
        \hline
        0&2&-&\\
        \hline
    \end{tabular}
\end{center}


\section*{verkefni 1.3.13}
í þessu verkefni er það aðeins a) sem hægt er að framkvæma með queue aðgerðum, þetta er vegna þess að það er enqueue-að heiltölunum í réttri röð, $[0,1,2,3,4,5,6,7,8,9]$, inn í röðina og þar sem dequeue skilar elsta staki sem þyrfti þá í öllum þessum tilfellum að vera $0$

\end{document}